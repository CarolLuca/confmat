\documentclass{article}
\usepackage{graphicx} % Required for inserting images
\usepackage{tikz}
\usepackage{listings}
\usepackage{color}
\usepackage{amsmath}

\usepackage{circuitikz}
\usepackage{color}

\definecolor{dkgreen}{rgb}{0,0.6,0}
\definecolor{gray}{rgb}{0.5,0.5,0.5}
\definecolor{mauve}{rgb}{0.58,0,0.82}

\usetikzlibrary{arrows}
\usepackage{amssymb}
\title{Google Matrix}
\author{adrian.ariton0 \\ carol.luca}
\date{June 2023}

\begin{document}

\maketitle

\section{Introduction}
\subsection{Power Method}

    \textbf{eigenvector.m}
\lstset{frame=tb,
  language=Octave,
  aboveskip=3mm,
  belowskip=3mm,
  showstringspaces=false,
  columns=flexible,
  basicstyle={\small\ttfamily},
  numbers=none,
  numberstyle=\tiny\color{gray},
  keywordstyle=\color{blue},
  commentstyle=\color{dkgreen},
  stringstyle=\color{mauve},
  breaklines=true,
  breakatwhitespace=true,
  tabsize=3
}
    \begin{lstlisting}
    function retval = eigenvector (stochastic_matrix, precision)
      n = size(stochastic_matrix,1);
      tol = 10 ^ (-precision);
      output_precision(precision);
      vector = ones(n, 1) * 1 / n;
      old_vector = zeros(n, 1);
      iter = 1;
    
      while norm(old_vector - vector) > tol
        old_vector = vector;
        printf("A ^ %d * ", iter)
        vector = stochastic_matrix * vector
        iter = iter + 1;
        norm(old_vector - vector);
      endwhile
    
      printf("\n-------------------------");
      printf("\nStabilized on %dth iteration\n", iter-1);
    
      retval = vector;
    endfunction
    \end{lstlisting}  
    \subsection{Directed Link Graph}
    \begin{center}
    \begin{tikzpicture}
    
    \tikzset{vertex/.style = {shape=circle,draw,minimum size=1.5em}}
    \tikzset{edge/.style = {->,> = latex'}}
    % vertices
    \node[vertex,minimum size=1cm] (5) at  (0,0) {$5$};
    \node[vertex,minimum size=0.3cm] (1) at  (0,6) {$1$};
    \node[vertex,minimum size=1cm] (2) at  (5,8) {$2$};
    \node[vertex,minimum size=2cm] (4) at  (5,-2) {$4$};
    \node[vertex,minimum size=0.2cm] (3) at (9, 3) {$3$};
    %edges

    
    \draw[edge] (1)  to (2);
    \draw[edge] (2)  to[bend left] (4);
    \draw[edge] (1)  to (4);

    \draw[edge] (4)  to[bend left] (2);
    \draw[edge] (3)  to (4);

        
    \draw[edge] (4)  to (5);
    \draw[edge] (3)  to (5);
    \draw[edge] (5)  to (1);
    
    \end{tikzpicture}
    \end{center}
    $$A = \begin{bmatrix}0 & 0 & 0 & 0 & 1 \\ \frac{1}{3} & 0 & 0 & \frac{1}{2} & 0 \\ \frac{1}{3} & 0 & 0 & 0& 0\\ \frac{1}{3} & 1 & \frac{1}{2} & 0 & 0\\ 0 & 0 & \frac{1}{2} & \frac{1}{2} & 0\end{bmatrix}$$

$$v = \begin{bmatrix}
  0.2\\
  0.2\\
  0.2\\
  0.2\\
  0.2\\
  \end{bmatrix}
  $$
  
      $$A^{1.0} \cdot v = \begin{bmatrix}
  0.2\\
  0.1666\\
  0.0666\\
  0.3666\\
  0.2
  \end{bmatrix}
  $$
  
  $$A^{2.0} \cdot v = \begin{bmatrix}
  0.2\\
  0.25\\
  0.0666\\
  0.2666\\
  0.2166
  \end{bmatrix}
  $$
  
  $$A^{3.0} \cdot v = \begin{bmatrix}
  0.2166\\
  0.2\\
  0.0666\\
  0.35\\
  0.1666
  \end{bmatrix}
  $$
  
  $$A^{4.0} \cdot v = \begin{bmatrix}
  0.1666\\
  0.2472\\
  0.0722\\
  0.3055\\
  0.2083
  \end{bmatrix}
  $$
  
  $$A^{5.0} \cdot v = \begin{bmatrix}
  0.2083\\
  0.2083\\
  0.0555\\
  0.3388\\
  0.1888
  \end{bmatrix}
  $$
  
  $$A^{6.0} \cdot v = \begin{bmatrix}
  0.1888\\
  0.2388\\
  0.0694\\
  0.3055\\
  0.1972
  \end{bmatrix}
  $$
  
  $$A^{7.0} \cdot v = \begin{bmatrix}
  0.1972\\
  0.2157\\
  0.0629\\
  0.3365\\
  0.1875
  \end{bmatrix}
  $$
  
  $$A^{8.0} \cdot v = \begin{bmatrix}
  0.1875\\
  0.234\\
  0.0657\\
  0.3129\\
  0.1997
  \end{bmatrix}
  $$
  
  $$A^{9.0} \cdot v = \begin{bmatrix}
  0.1997\\
  0.2189\\
  0.0625\\
  0.3293\\
  0.1893
  \end{bmatrix}
  $$
  
  $$A^{10.0} \cdot v = \begin{bmatrix}
  0.1893\\
  0.2312\\
  0.0665\\
  0.3168\\
  0.1959
  \end{bmatrix}
  $$
  
  $$A^{11.0} \cdot v = \begin{bmatrix}
  0.1959\\
  0.2215\\
  0.0631\\
  0.3277\\
  0.1917
  \end{bmatrix}
  $$
  
  $$A^{12.0} \cdot v = \begin{bmatrix}
  0.1917\\
  0.2291\\
  0.0653\\
  0.3184\\
  0.1954
  \end{bmatrix}
  $$
  
  $$A^{13.0} \cdot v = \begin{bmatrix}
  0.1954\\
  0.2231\\
  0.0639\\
  0.3257\\
  0.1918
  \end{bmatrix}
  $$
  
  $$A^{14.0} \cdot v = \begin{bmatrix}
  0.1918\\
  0.2279\\
  0.0651\\
  0.3201\\
  0.1948
  \end{bmatrix}
  $$
  
  $$A^{15.0} \cdot v = \begin{bmatrix}
  0.1948\\
  0.224\\
  0.0639\\
  0.3245\\
  0.1926
  \end{bmatrix}
  $$
  
  $$A^{16.0} \cdot v = \begin{bmatrix}
  0.1926\\
  0.2271\\
  0.0649\\
  0.3209\\
  0.1942
  \end{bmatrix}
  $$
  
  $$A^{17.0} \cdot v = \begin{bmatrix}
  0.1942\\
  0.2247\\
  0.0642\\
  0.3238\\
  0.1929
  \end{bmatrix}
  $$
  
  $$A^{18.0} \cdot v = \begin{bmatrix}
  0.1929\\
  0.2266\\
  0.0647\\
  0.3215\\
  0.194
  \end{bmatrix}
  $$
  
  $$A^{19.0} \cdot v = \begin{bmatrix}
  0.194\\
  0.225\\
  0.0643\\
  0.3233\\
  0.1931
  \end{bmatrix}
  $$
  
  $$A^{20.0} \cdot v = \begin{bmatrix}
  0.1931\\
  0.2263\\
  0.0646\\
  0.3219\\
  0.1938
  \end{bmatrix}
  $$
  
  $$A^{21.0} \cdot v = \begin{bmatrix}
  0.1938\\
  0.2253\\
  0.0643\\
  0.3231\\
  0.1933
  \end{bmatrix}
  $$
  
  $$A^{22.0} \cdot v = \begin{bmatrix}
  0.1933\\
  0.2261\\
  0.0646\\
  0.3221\\
  0.1937
  \end{bmatrix}
  $$
  
  $$A^{23.0} \cdot v = \begin{bmatrix}
  0.1937\\
  0.2255\\
  0.0644\\
  0.3229\\
  0.1933
  \end{bmatrix}
  $$
  
  $$A^{24.0} \cdot v = \begin{bmatrix}
  0.1933\\
  0.226\\
  0.0645\\
  0.3223\\
  0.1936
  \end{bmatrix}
  $$
  

-------------------------\\
Stabilized on 24th iteration




    %%%%%%%%%%

    \subsection{Undirected Link Graph}
    \begin{center}
    \begin{tikzpicture}
    
    \tikzset{vertex/.style = {shape=circle,draw,minimum size=1.5em}}
    \tikzset{edge/.style = {->,> = latex'}}
    % vertices
    \node[vertex,minimum size=1cm] (5) at  (0,0) {$5$};
    \node[vertex,minimum size=1cm] (1) at  (0,6) {$1$};
    \node[vertex,minimum size=1cm] (2) at  (5,8) {$2$};
    \node[vertex,minimum size=0.5cm] (4) at  (5,-2) {$4$};
    \node[vertex,minimum size=1cm] (3) at (9, 3) {$3$};
    %edges

    
    \draw[edge] (1)  to (2);
    \draw[edge] (2)  to (1);
    
    \draw[edge] (2)  to (4);
    \draw[edge] (4)  to (2);
        
    \draw[edge] (4)  to (5);
    \draw[edge] (5)  to (4);
        
    \draw[edge] (1)  to (3);
    \draw[edge] (3)  to (1);
        
    \draw[edge] (2)  to (3);
    \draw[edge] (3)  to (2);
        
    \draw[edge] (5)  to (3);
    \draw[edge] (3)  to (5);
        
    \draw[edge] (1)  to (5);
    \draw[edge] (5)  to (1);
    
    \end{tikzpicture}
    \end{center}
    $$A = \begin{bmatrix}0 & \frac{1}{3} & \frac{1}{3} & 0 & \frac{1}{3} \\
    \frac{1}{3} & 0 & \frac{1}{3} & \frac{1}{2} & 0 \\
    \frac{1}{3} & \frac{1}{3} & 0 & 0& \frac{1}{3}\\
    0 & \frac{1}{3} & 0 & 0 & \frac{1}{3}\\
    \frac{1}{3} & 0 & \frac{1}{3} & \frac{1}{2} & 0\end{bmatrix}$$

    $$v = \begin{bmatrix}
      0.2\\
      0.2\\
      0.2\\
      0.2\\
      0.2\\
      \end{bmatrix}
      $$
      
          $$A^{1.0} \cdot v = \begin{bmatrix}
      0.2\\
      0.2333\\
      0.2\\
      0.1333\\
      0.2333
      \end{bmatrix}
      $$
      
      $$A^{2.0} \cdot v = \begin{bmatrix}
      0.2222\\
      0.2\\
      0.2222\\
      0.1555\\
      0.2
      \end{bmatrix}
      $$
      
      $$A^{3.0} \cdot v = \begin{bmatrix}
      0.2074\\
      0.2259\\
      0.2074\\
      0.1333\\
      0.2259
      \end{bmatrix}
      $$
      
      $$A^{4.0} \cdot v = \begin{bmatrix}
      0.2197\\
      0.2049\\
      0.2197\\
      0.1506\\
      0.2049
      \end{bmatrix}
      $$
      
      $$A^{5.0} \cdot v = \begin{bmatrix}
      0.2098\\
      0.2218\\
      0.2098\\
      0.1366\\
      0.2218
      \end{bmatrix}
      $$
      
      $$A^{6.0} \cdot v = \begin{bmatrix}
      0.2178\\
      0.2082\\
      0.2178\\
      0.1478\\
      0.2082
      \end{bmatrix}
      $$
      
      $$A^{7.0} \cdot v = \begin{bmatrix}
      0.2114\\
      0.2191\\
      0.2114\\
      0.1388\\
      0.2191
      \end{bmatrix}
      $$
      
      $$A^{8.0} \cdot v = \begin{bmatrix}
      0.2165\\
      0.2103\\
      0.2165\\
      0.1461\\
      0.2103
      \end{bmatrix}
      $$
      
      $$A^{9.0} \cdot v = \begin{bmatrix}
      0.2124\\
      0.2174\\
      0.2124\\
      0.1402\\
      0.2174
      \end{bmatrix}
      $$
      
      $$A^{10.0} \cdot v = \begin{bmatrix}
      0.2157\\
      0.2117\\
      0.2157\\
      0.1449\\
      0.2117
      \end{bmatrix}
      $$
      
      $$A^{11.0} \cdot v = \begin{bmatrix}
      0.213\\
      0.2163\\
      0.213\\
      0.1411\\
      0.2163
      \end{bmatrix}
      $$
      
      $$A^{12.0} \cdot v = \begin{bmatrix}
      0.2152\\
      0.2126\\
      0.2152\\
      0.1442\\
      0.2126
      \end{bmatrix}
      $$
      
      $$A^{13.0} \cdot v = \begin{bmatrix}
      0.2135\\
      0.2156\\
      0.2135\\
      0.1417\\
      0.2156
      \end{bmatrix}
      $$
      
      $$A^{14.0} \cdot v = \begin{bmatrix}
      0.2149\\
      0.2132\\
      0.2149\\
      0.1437\\
      0.2132
      \end{bmatrix}
      $$
      
      $$A^{15.0} \cdot v = \begin{bmatrix}
      0.2137\\
      0.2151\\
      0.2137\\
      0.1421\\
      0.2151
      \end{bmatrix}
      $$
      
      $$A^{16.0} \cdot v = \begin{bmatrix}
      0.2146\\
      0.2135\\
      0.2146\\
      0.1434\\
      0.2135
      \end{bmatrix}
      $$
      
      $$A^{17.0} \cdot v = \begin{bmatrix}
      0.2139\\
      0.2148\\
      0.2139\\
      0.1423\\
      0.2148
      \end{bmatrix}
      $$
      
      $$A^{18.0} \cdot v = \begin{bmatrix}
      0.2145\\
      0.2138\\
      0.2145\\
      0.1432\\
      0.2138
      \end{bmatrix}
      $$
      
      $$A^{19.0} \cdot v = \begin{bmatrix}
      0.214\\
      0.2146\\
      0.214\\
      0.1425\\
      0.2146
      \end{bmatrix}
      $$
      
      $$A^{20.0} \cdot v = \begin{bmatrix}
      0.2144\\
      0.2139\\
      0.2144\\
      0.143\\
      0.2139
      \end{bmatrix}
      $$
      
      $$A^{21.0} \cdot v = \begin{bmatrix}
      0.2141\\
      0.2145\\
      0.2141\\
      0.1426\\
      0.2145
      \end{bmatrix}
      $$

-------------------------\\
Stabilized on 21th iteration


\section{Metoda Puterii}


Let $A \in \mathbb{R}^{n \times n}$ be the matrix for which we consider the spectrum of eignevalues $\lambda(A) = \{\lambda_1, \lambda_2, \ldots, \lambda_n\}$ and the set of eigenvectors $X = \{x_1, x_2, \ldots, x_n\}$, whose Euclidean norm is one, of matrix $A$. On top of that, we assume that $|\lambda_1| \geq |\lambda_2| \geq \ldots \geq |\lambda_n|$. Let $y \in \mathbb{C}^n$ be an eigenvector of Euclidean norm one having a non-zero component with the direction of the eigenvector $x_1 \in X$. \\


The direct power method assumes the defining of the sequences $(y(k))_{k \in \mathbb{N}}$ and $(\lambda(k))_{k \in \mathbb{N}}$ as follows: \\
----------------------------------------------- \\
$y(0) = y$ \\
For $k = 1, 2, \ldots, \text{{max}}$:\\
\hspace{1cm}$z \leftarrow A \cdot y(k-1)$\\
\hspace{1cm}$y(k) \leftarrow z / \|z\|_2$\\
\hspace{1cm}$\lambda(k) \leftarrow (y(k))^T A y(k)$ \\
------------------------------------------------

Therefore, the numerical sequence $(\lambda(k))_{k \in \mathbb{N}}$ converges to the dominant eigenvalue, while in the meantime the vectorial sequence $(y(k))_{k \in \mathbb{N}}$ converges to the unitary eigenvector associated to the dominant eigenvalue. The direct power method converges for any initially chosen vector $y$.

\section{Applications of Google PageRank}

Google PageRank, initially developed as a web page ranking algorithm, has found various applications beyond its original purpose. In this section, we explore both web-based and non-web-based applications of PageRank.

\subsection{Web-Based Applications}

\subsubsection{Web Search Ranking}

One of the most prominent applications of PageRank is in web search ranking. Google's search engine utilizes PageRank as a crucial component in determining the relevance and ranking of web pages. By considering the link structure of the web, PageRank helps prioritize pages based on their importance and authority, providing users with more relevant search results.

\subsubsection{Recommendation Systems}

PageRank has been employed in recommendation systems to suggest relevant and personalized content to users. By analyzing the link structure and user behavior, PageRank can identify influential pages or items and recommend them to users based on their interests or browsing history. This application is particularly valuable in e-commerce platforms, news aggregators, and social media networks.

\subsubsection{Social Network Analysis}

PageRank algorithms have been adapted to analyze and understand social networks. By treating individuals as nodes and social connections as edges, PageRank can identify influential individuals or communities within a network. This analysis helps reveal key players, opinion leaders, and communities of interest, enabling targeted marketing, information diffusion studies, and social network visualization.

\subsection{Non-Web-Based Applications}

\subsubsection{Academic Citations and Research Impact}

PageRank concepts have been applied to academic citations and measuring research impact. By considering citations as links between academic papers, PageRank algorithms can assess the influence and importance of scholarly articles. This application aids in identifying influential researchers, journals, and conferences, and assists in evaluating research impact and collaborations.

\subsubsection{Image Analysis and Object Recognition}

PageRank-inspired algorithms have been employed in image analysis and object recognition tasks. By treating images as nodes and visual similarities as edges, PageRank-based approaches can identify important images or objects within a large collection. This application finds use in image search engines, recommendation systems for visual content, and image clustering.

\subsubsection{Data Mining and Knowledge Discovery}

PageRank algorithms have been applied in data mining and knowledge discovery processes. By modeling data relationships as a graph, PageRank can identify important entities, patterns, or clusters within datasets. This application aids in identifying influential users in social networks, detecting anomalies in network traffic, and uncovering hidden relationships in large-scale datasets.

In conclusion, Google PageRank has proven to be a versatile algorithm with applications extending beyond web-based ranking. Its concepts and variations have found utility in recommendation systems, social network analysis, academic research, image analysis, data mining, and more. These applications highlight the broad impact and significance of PageRank in various domains.


\end{document}
